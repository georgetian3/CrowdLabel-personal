\section{第一轮需求获取} \label{sec:interview1}

\begin{enumerate}
    \item \textbf{已有的竟品有什么不足之处?甲方希望此项目增加什么其他项目没有的功能?}
    \begin{itemize}
        \item 众包平台太过于冗余,没有一个明确的介绍
        \item 竟品的导航不清楚、太繁琐了
        \item 功能实现的话,别的平台有的需要在做一个(做一个类似的)
    \end{itemize}
    \item \textbf{本项目为营利性还是非营利性?若营利性,如何获取收入?(广告、收部分奖金等)}
    
    可以自行设置,最好是营利性

    \item \textbf{目标用户是谁?(清华学生、高校学生、国内、全球等)}

    都没有特定的需求,谁都可以。

    \item \textbf{注册时需要用户的什么信息?(用户名、密码、手机号、邮箱、年龄、性别等)}
    
    用户名、密码、邮箱、手机号、支付报酬的途径(支付宝、微信、银行卡)

    \item \textbf{对于任务接收者是否需要额外的信息?}
    
    不
    
    \item \textbf{发布者对接收者采用什么样的报酬形式?虚拟或真实货币?是否需要关联微信/支付宝?}
    
    报酬方式:平台中间账户进行打钱操作(困难),或者是任务发布者直接发钱。报酬通过:积分兑换,直接获得现金也行,没有严格的要求
    
    \item \textbf{平台运行的过程中需要记录发布者以及接收者的什么信息?比如接收者的准确率、完成任务历史,发布者的任务发布历史、接收者对发布者的评价。这些数据可以让平台辨认不认真/用脚本的接收者,也可以让接收者判断是否接收某个发布者的任务。}
    
    任务发布历史,接受者的历史信息如果有时间去弄的话,可以

    \item \textbf{需要哪些打标形式?(单选、多选、画框、排序等)}
    
    
    \item \textbf{需要被打标的数据类型?(数字、文字、图片、音频、视频等)}

    打标就四种就行:按照项目介绍ppt上的那几种都做出来就行
    
    \item \textbf{任务是否需要标签?(情绪识别、物体识别等)。标签是否让发布者自定义?是否需要管理员整理任务标签?比如 stackoverflow 问题的标签。}
    
    任务可以有标签,但事实上同一类打标情绪任务都是类似的。标签可以由发布者定义,类别标签或者任务标签都可以。最好是咱们让他们从一个标签库中搜索,省的出现问题。

    \item \textbf{作为接收方,希望任务发送方提供怎样的任务信息?(任务描述、奖金数额等)}
    
    接收方,提供案例

    \item \textbf{作为发送方,是否希望测试接收方用户的资质?希望如何测试?目前的想法:为每个数据集嵌入一些正确的标点,根据正确率数据计算积分。正确的标点希望谁来打?(任务发送方 vs 其他等级高的接收方用户)}
    
    发送方,用户资质测试分为两类:刚进网站,进行预设任务(每一种类型做一个)。另一种是发布者提供资质测试的内容。(测试在每一次任务之前)
	报酬是通过人工检验后再发放。

    \item \textbf{作为发送方,是否希望针对不同任务筛选可以参与的人员类型?比如任务发布者可能只要住在xx市,拥有xx学位的人完成任务。这些信息在用户注册时填,还是接收任务前填?}
    \item \textbf{希望以哪种形式使用产品?比如网页/桌面应用/小程序/单独手机应用。}
    \item \textbf{需要支持哪些操作系统?}
    \item 
    网页端就行。

    \item \textbf{对新手引导有何要求或期待?}
    
    新手引导:流程上的操作就行,足够清晰就行。

    \item \textbf{对界面/交互风格有何要求或期待?}

    界面:没有具体的要求,按照自己设计的来就行。可以参考:\url{https://github.com/heartexlabs/label-studio}

\end{enumerate}