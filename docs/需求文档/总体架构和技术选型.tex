\section{总体架构和技术选型}

此项目以网络应用的形式实现。网络应用的优点在于前端开发者只需要编写一个与平台独立的前端页面就可以在大部分可以运行浏览器的平台上使用。此外,此项目的前端没有涉及到高性能功能(图像渲染、多进程运算等等),则网络应用足以满足我们的性能需求。考虑到中国用户,由于微信小程序的前端技术类似于 HTML+CSS+JS,所以如果以后有开发小程序的需求,那迁移过程会相当简单。类似的,如果需要开发桌面或单独手机应用,可以使用 WebView 等框架。

后端将采用 Python 3,因为它具有丰富的数据处理以及网络应用开发包,并且此项目的性能瓶颈将会是输入输出而非计算速度。对于网络应用框架我们从 Django 和 Flask 中选择。 Django 更加适用于含有多个应用的项目。它也默认提供若干功能,但我们的开发团队偏向于简约的开发方式,即如果需要某个功能再手动加入到项目中。因此我们选择了使用 Flask。

我们选择使用 MySQL 作为数据库管理系统,因为它满足我们的性能要求、存在 Python 接口、并且我们的数据库管理人员对它有丰富的经验。