\section{用户痛点}

通过对 Amazon Mechanical Turk、百度众包、蚂蚁众包、LabelStudio 等多家竞品的体验与分析,我们对任务发布方和任务接收方分别总结了以下痛点,并简述了我们平台相应的解决方案。

\subsection{任务发布方用户痛点}

\begin{enumerate}
    \item 发布方入驻门槛高。根据我们对竞品的观察,大部分竞品需要对任务发布方进行较为严格的资质认证,甚至部分竞品需要经过平台邀请才允许发布方入驻。这严重影响了发布方用户的使用资格与意愿。因此,我们平台将不对发布方用户进行资质审核,任何人都可以作为任务发布方入驻。由于众包任务的劳务费是由任务发布方来出,所以我们相信,有了金钱的激励,绝大多熟发布方都会对自己发布的任务质量进行自我监督。
    \item 收集的数据不可靠或正确率较低。在许多竞品中,接收方用户都只需在入驻时完成一次测试就可以接任务了,而他们在这些后续任务中的表现将不再经过审核。因此,可能会出现接收方胡乱打标、但依然能赚到劳务费的情况。对此,我们为每项任务都设计了审核机制,只有当任务接收方的打标数据通过审核后,该任务的数据才会到达发送方、该任务的费用也才会到达接收方。
    \item 界面不够美观。我们观察到许多竞品的UI设计都较为粗糙古板,且看上去“年久失修”,这一定程度上降低了用户的使用意愿。在我们的产品中,我们将贯穿简约、大气的UI设计,并使用了醒目活泼的紫色和干净的白色作为主题色调,提升用户的视觉体验。
\end{enumerate}

\subsection{任务接收方用户痛点}

\begin{enumerate}
    \item 引导交互不友好,新手不知道该怎么使用。在许多竞品中,页面都存在着广告、社交功能、礼品功能等干扰或冗余元素,外加交互引导不够友好、注册成功后没有明确的指示或入口,使得新人用户很难上手。对此,我们设计了极其简约的网站构成,并对必要的用户操作做了明确的指示。比如,我们会在新用户注册后以强制弹出的方式提醒用户完成一份简单的入驻测试,完成测试后方可浏览并接收众包任务。
    \item 界面不够美观,详见上方第三点。
    \item 接收方的新人入驻门槛同样很高。在许多竞品中,任务接收方也需要经过人工身份审核或大量打标测试才可入驻。我们的产品去除了这些繁琐的门槛,用户只需绑定一个真实可用的邮箱(通过邮箱验证码验证)、并完成一份小体量的打标测试即可。至于用户之后的打标表现,我们会通过每一个任务自身的审核机制完成检验。
    \item 提现不方便。大部分竞品都使用了积分兑换礼物的形式,而不支持直接提现、或就算支持提现也只提供了向银行卡提现的选项,不符合中国消费者的习惯。我们的平台将同时提供微信、支付宝和银行卡三个提现选项,并允许用户直接提现,而不是引导用户去消费其他的产品。
\end{enumerate}